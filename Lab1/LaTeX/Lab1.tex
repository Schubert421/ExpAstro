% Autor: Manuel Lippert
% Physikalisches Praktikum

% Main-Datei für die Auswertung in TeX

% Struktur:
% Jedes Kapitel hat einen Input-File. Um Merge-Konflikte zu verhindern wird angeraten für jede 
% Datei eine eigene Tex Datei zu machen und sie im jeweiligen Kapitel zu importieren. Die in
% Input-Struktur dient zur besseren Übersicht und für mögliche Ordner, welche hier vorhanden sind. Die Zahlen vor den 
% Ordner dient zur Ordnung der einzelnen tex-Files nach Kapiteln


% Packages
\documentclass[paper=a4,bibliography=totoc,BCOR=10mm,twoside,numbers=noenddot,fontsize=11pt]{scrreprt}
\usepackage[ngerman]{babel}
\usepackage[T1]{fontenc}
\usepackage[latin1, utf8]{inputenc} %ä, ö, ü inbegriffen
\usepackage[babel,german=quotes]{csquotes} %For Quotes
\usepackage{lmodern}
\usepackage{graphicx}
\usepackage{nicefrac}
\usepackage{fancyvrb}
\usepackage{amsmath,amssymb,amstext}
\usepackage{siunitx}
\usepackage{url}
\usepackage{natbib}
\usepackage{microtype}
\usepackage[format=plain]{caption}
\usepackage{physics}
\usepackage{titleref} 

% Zusätzliche Packages
\usepackage{geometry} % Verändert Seitengeometrie
\usepackage{anyfontsize} % Alle Schriftgrößen möglich machen
\usepackage[table]{xcolor} % Farbliche Gestaltung Tabellen
\usepackage{ifthen} % Für kompliziertere tex-Files
\usepackage[absolute,overlay]{textpos} %Textboxen
\usepackage{amsfonts} % Schriftarten
\usepackage{xstring} % Stringoperationen
\usepackage{tikz} % Zeichnungen
\usepackage{pdfpages} % Import von pdfs (Protokolle)
\usepackage{hyperref} % Verlinkungen im Dokument
\usepackage{siunitx}  % Zahlen und Einheiten für NW
\sisetup{locale = DE} 

% Abschnittseinrückung und -abstand
% Die folgenden Zeilen sollen möglichst nicht verändert werden
\parindent 0.0cm
\parskip 0.8ex plus 0.5ex minus 0.5ex

% Anzahl und Größe von Gleitobjekten
% maximal 2 Objekte oben und unten
% erlaubt auch größere Bilder, welche die ganze Seite benötigen
% Die folgenden Zeilen sollen möglichst nicht verändert werden
\setcounter{bottomnumber}{2}
\setcounter{topnumber}{2}
\renewcommand{\bottomfraction}{1.}
\renewcommand{\topfraction}{1.}
\renewcommand{\textfraction}{0.}

%\sc und \bc veraltet. Daher: (20.09.2018)
\DeclareOldFontCommand{\sc}{\normalfont\scshape}{\@nomath\sc}
\DeclareOldFontCommand{\bf}{\normalfont\scshape}{\textbf}

% Verschiedenes
\pagestyle{headings}          % Der Seitenstil sollte möglichst nicht verändert werden
\graphicspath{{./Bilder/}}    % Der Pfad für die Abbildungen Abbildungen wird gesetzt
\VerbatimFootnotes            % \verb etc. auch in \footnotes mφglich

% Funktionen
\newcommand\tab[1][1cm]{\hspace*{#1}}
\newcommand{\vect}[1]{\boldsymbol{\mathbf{#1}}}
\newcolumntype{g}{>{\columncolor[rgb]{ .741,  .843,  .933}}l}
% Tiefgestellte Zahlen nicht kursiv
\catcode`_=\active
\newcommand_[1]{\ensuremath{\sb{\mathrm{#1}}}}

\begin{document}

    \nonfrenchspacing

    % 0. Kapitel Cover
    % 0. Cover

% Hier sind nur die Variablen und der Abschnitt Informationen (unten) zu bearbeiten der REst läuft automatisch ab (z.b Farbenänderung)

% Noch abänderbar nur ein Vorschlag
\newgeometry{top=30mm, bottom=20mm, inner=20mm, outer=20mm}
\thispagestyle{empty}

% Colors (Notability Colors)
\definecolor{Notablue}{HTML}{3498DB}		
\definecolor{Notared}{HTML}{CF366C}			
\definecolor{Notagreen}{HTML}{19B092}		
\definecolor{Notaorange}{HTML}{FA9D00}		
\definecolor{Notagrey}{HTML}{969696}		
\definecolor{Notalavendel}{HTML}{9DBBD8}	

% Boolean by default false. Für Absatz in der Überschrift
\newboolean{twoRows}
\newboolean{symbol}

% Funktions
\makeatletter
   \def\vhrulefill#1{\leavevmode\leaders\hrule\@height#1\hfill \kern\z@}
\makeatother
\newcommand*\ruleline[1]{\par\noindent\raisebox{.8ex}{\makebox[\linewidth]{\vhrulefill{\linethickness}\hspace{1ex}\raisebox{-.8ex}{#1}\hspace{1ex}\vhrulefill{\linethickness}}}}

% Variables
\def\schriftgrosse{70}
\def\linethickness{1,5pt}

\def\farbe{black}
\def\fach{Fach}
\def\name{Teilnehmer}
\def\titel{Titel} % Absatz mit \\[0,5cm]
\def\bottom{Fachsemester}
\def\datum{Datum}
\def\platz{NWII | RaumNr.}
\def\betreuer{Betreuer}

% Für Fortgeschrittenen Praktikum siehe unten
\def\teilnehmerm{Teilnehmer1}
\def\emailm{teilnehmer1@e-mail.de}
\def\teilnehmerp{Teilnehmer2}
\def\emailp{teilnehmer2@e-mail.de}

% Für Grundpraktikum siehe unten
\def\auswertp{Teilnehmer1}
\def\messp{Teilnehmer2}
\def\protop{Teilnehmer3}

\def\groupnr{XY}

\begin{titlepage}
			
	\centering
	{\LARGE \sffamily {\textbf{\bottom}\par}}
	\vspace{2,5cm}
    {\fontsize{30}{0}\sffamily\ruleline{\textcolor{\farbe}{\textbf{\fach}}}\par}
    \vspace{6cm}
	{\Large\sffamily \ruleline{\name}\par}
		
	\IfSubStr {\titel} {\\[0,5cm]} {\setboolean{twoRows}{true}} {\setboolean{twoRows}{false}}
	
	\ifthenelse{\boolean{twoRows}}
		{
			\begin{textblock*}{21cm}(0cm,8cm) % {block width} (coords), centered		
				{\fontsize{\schriftgrosse}{0}\sffamily\textcolor{\farbe}{\textbf{\titel}}\par}
			\end{textblock*}
		}
		{
			\begin{textblock*}{21cm}(0cm,9cm) % {block width} (coords), centered		
				{\fontsize{\schriftgrosse}{0}\sffamily\textcolor{\farbe}{\textbf{\titel}}\par}
			\end{textblock*} 
		}
	
	% Choose Logo
	\ifthenelse {\equal{\farbe}{Notared}} {\def\logo{Bilder/Logo/UniBTNotared}}
		{\ifthenelse {\equal{\farbe}{Notagreen}} {\def\logo{Bilder/Logo/UniBTNotagreen}}
			{\ifthenelse {\equal{\farbe}{Notablue}} {\def\logo{Bilder/Logo/UniBTNotablue}}
				{\ifthenelse {\equal{\farbe}{Notaorange}} {\def\logo{Bilder/Logo/UniBTNotaorange}}
					{\ifthenelse {\equal{\farbe}{Notagrey}} {\def\logo{Bilder/Logo/UniBTNotagrey}}
						{\ifthenelse {\equal{\farbe}{Notalavendel}} {\def\logo{Bilder/Logo/UniBTNotalavendel}}	
							{\ifthenelse {\equal{\farbe}{black}} {\def\logo{Bilder/Logo/UniBT}}	
								{\def\logo{noLogo}}
							}
						}
					}
				}
			}
		}	

	\IfSubStr{\logo}{noLogo}{\setboolean{symbol}{false}}{\setboolean{symbol}{true}}
	
	% Gruppe
	\vspace{10cm}
	{\large\sffamily{Gruppe \groupnr}}
	
	%Logo
	\vfill

	\ifthenelse{\boolean{symbol}}
		{
			\begin{figure}[h]
			\begin{center}
				
				\includegraphics[width=2cm]{\logo}
				
			\end{center}
			\end{figure}
		}
	
\end{titlepage}

\restoregeometry

% Information
\chapter*{Informationen}
\label{chap:info}

\begin{tabular}{l l}

	{\textbf{Versuchstag}} \hspace{1cm} & \hspace{1cm} {\datum}\\[0,2cm]
	{\textbf{Versuchsplatz}} \hspace{1cm} & \hspace{1cm} {\platz}\\[0,2cm]
	{\textbf{Betreuer}} \hspace{1cm} & \hspace{1cm} {\betreuer}\\[1,2cm]
	{\textbf{Gruppen Nr.}} \hspace{1cm} & \hspace{1cm} {\groupnr}\\[0.2cm]
	% Für Fortgeschittenenen Praktikum
	%{\textbf{Teilnehmer}} \hspace{1cm} & \hspace{1cm} {\teilnehmerm~(\emailm)}\\[0.2cm]
	%					  \hspace{1cm} & \hspace{1cm} {\teilnehmerp~(\emailp)}\\[0.2cm]
	% Für Grundpraktikum
	{\textbf{Auswertperson}} \hspace{1cm} & \hspace{1cm} {\auswertp}\\[0.2cm]
	{\textbf{Messperson}} \hspace{1cm} & \hspace{1cm} {\messp}\\[0.2cm]
	{\textbf{Protokollperson}} \hspace{1cm} & \hspace{1cm} {\protop}\\[0.2cm]

\end{tabular}

    \thispagestyle{empty}
    \cleardoublepage
    \tableofcontents
    \cleardoublepage

    % 1. Kapitel Einleitung
    % 1. Einleitung

\chapter{Einleitung}
\label{chap:einleitung}

% Text

    % 2.Kapitel Fragen zur Vorbereitung
    % 2. Fragen zur Vorbereitung

\chapter{Fragen zur Vorbereitung}
\label{chap:fvz}

% Text

% Input der Teilaufgaben je nach Produktion der Nebendateien ohne Ordner
% Teilaufgabe X

\section{Teilaufgabe X}

% etc.

    % 3.Kapitel Protokoll
    % 3. Protokoll

% Variables
\def\skalierung{0.65}

\chapter{Messprotokoll}
\label{chap:protokoll}

Das Messprotokoll wurde am Versuchstag handschriftlich erstellt und hier als
PDF-Datei eingefügt. Dabei wurden Durchführung und Aufbau schon vorher in dieses
Dokument beschrieben, je nachdem.

% Einbindung des Protokolls als pdf (mit Seitenzahl etc.)
% Erste Seite mit Überschrift
%\includepdf[pages = 1, landscape = false, nup = 1x1, scale = \skalierung , pagecommand={\thispagestyle{empty}\chapter{Protokoll}}]
%            {03-Protokoll/Protokoll.pdf}
% Restliche Seiten richtig skaliert
%\includepdf[pages = -, landscape = false, nup = 1x1, scale = \skalierung , pagecommand={}]
%            {03-Protokoll/Protokoll.pdf}

    % 4.Kapitel Versuchsauswertung
    % 4. Versuchsauswertung

\chapter{Auswertung und Diskussion}
\label{chap:versuchsauswertung}

% Text

% Input der Teilauswertung je nach Produktion der Nebendateien ohne Ordner
% Teilauswertung X

\section{Teilauswertung X}

% etc.

    % 5.Kapitel Fazit
    % 5. Einleitung

\chapter{Fazit}
\label{chap:fazit}


% Text

    % Anhang
    % Anhang

\appendix

% Text

% Charlotte Geiger - Manuel Lippert - Leonard Schatt
% Physikalisches Praktikum

% Anhang A

\chapter{Append A}
\label{chap:anhangA}

\section{Teilanhang X}


    % Literatur
    \bibliographystyle{Auswertung.bst}
    \nocite{*}
    \bibliography{Auswertung.bib}

\end{document}
